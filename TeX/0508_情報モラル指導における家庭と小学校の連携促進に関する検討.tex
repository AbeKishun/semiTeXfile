\documentclass[a4j,11pt]{jsarticle}
\usepackage[dvipdfmx]{graphicx}

\begin{document}
\begin{flushright} %右揃え
2018年5月8日

阿部 希駿
\end{flushright}

\begin{center}
\Large{情報モラル指導における家庭と小学校の
連携促進に関する検討}
\end{center}

\section{背景、目的}
\label{sec:mokuteki}
コンピュータの普及率、携帯電話の普及率が飛躍的に上昇している一方で、インターネット利用に関するモラルやマナーが定着しておらず、青少年を巻き込んだ事件が起きている。
しかしgooリサーチが調査した小学生のインターネット利用に関する調査では就学前からの利用が3割以上であるにも関わらずインターネット利用のルールを設けていない家庭が全体の37,8%であった。情報モラルの関連資料を調査したところ保護者が情報モラルに関する情報を入手するが容易ではない事を示した。また子どもがインターネットを利用する時間は学校よりも家庭が長い。

これらのことから学校が主体となり、計画的な情報モラルについての指導を進め、家庭とより緊密を図ることを目的とした研究を行なった。
\section{手段}
\label{sec:kihon}
保護者を対象とした情報モラルに関する意識調査を実施する。これを基に学校と家庭がより緊密な連携を図るためのプログラムを開発し、その有効性を検証する。また開発したプログラムを実際に学校で行い保護者向けの調査を中心にして実施状況や到達状況を評価し分析する。
 

\section{保護者への意識調査}
\subsection{実施方法}
\label{sec:kih}
20の調査項目について、重要度を6段階で評価させた。200人の保護者に回答用紙を配布し161人から回答が得られた。
またその結果を因子分析を行い抽出因子間での因果関係を調べた。

\subsection{評価結果}
アンケートより平均値が高かった6項目には天上効果が現れたため除外し、残りの14項目を用いて因子分析を行った。その結果、4つの因子解「学校との連絡」「保護者のモラル向上」「保護者の関与」「家庭での指導内容」を抽出した。

また「学校との連絡」と「保護者のモラル向上」が関連しあい、「保護者の関与」が発揮されて「家庭での指導内容」に影響するという因果モデルを仮定した。




\section{提携プログラム}
保護者のモラル向上
\begin{enumerate}
\item 保護者向け説明会を実施する。
\item 情報モラルの授業の公開
  \end{enumerate}

保護者のモラル向上
\begin{enumerate}
\item 参観授業の感想や家庭での状況を学校に知らせる。
  \end{enumerate}
  
 保護者の関与
\begin{enumerate}
\item 保護者会の開催
\item 保護者間の情報共有
  \end{enumerate}
  
家庭での指導内容
\begin{enumerate}
\item 子どもと保護者の話し合いによってルールを決める。
\item 家庭での具体的な指導を行う
  \end{enumerate}



\section{プログラム参加後の保護者への意識調査}
\subsection{実施方法}
\label{sec:kih}
プログラム実施前と同様の調査をプログラム参加者28名に対し行い、因子ごとにプログラム実施前と比較した。

\subsection{評価結果}
\begin{table}[htb]
  \centering
  \caption{プログラム前後の因子ごとの比較}
\begin{tabular}{lcrr}
 & プログラム前 & プログラム後 \\ % セルは&で区切る,次の行に行くには\\を入れ
保護者のモラル向上 & 5.25 & 4.55 \\
家庭での指導内容 & 5.89 & 5.70\\
保護者の関与 & 5.74 & 5.27 \\
学校への連絡 & 4.64 & 3.60
  \end{tabular}
\end{table}

結果は表1のようになり全ての因子においてプログラム前に対して有意に高い水準となった。

\section{まとめ}
\label{sec:mokuteki}
学校と家庭での連携プログラムにより。保護者の情報モラルの向上を検証する事ができた。
今後はこれに加えて児童の情報モラルの向上にどのような効果をもたらすかを客観的に検証する必要がある。

\section{参考文献}
\label{sec:bunken}
山本朋弘,清水康敬“情報モラル指導における家庭と小学校連携促進に関する検討”,『日本教育工学会論文誌』Vol32(2), pp.181-188,2008.
  

\end{document}