\documentclass[a4j,11pt]{jsarticle}
\usepackage[dvipdfmx]{graphicx}

\begin{document}
\begin{flushright} %右揃え
2018年5月15日

阿部 希駿
\end{flushright}

\begin{center}
\Large{スウェーデン小学校社会科の教科書を読む 1}
\end{center}

\section{背景}
\label{sec:mokuteki}
日本では「若者の政治離れ」や「政治意識の希薄化」が度々問題視されています.日本で2014年に行われた衆議院議員総選挙の投票率は52.7%30歳未満に限れば32.6%とかなり低くなっています.それに対してスウェーデンで同年行われた国政選挙の投票率は85.8%,30歳未満の投票率に限っても81.3%と高い水準です.

\section{原因}
内閣府が2013年に日本と諸外国7カ国の13〜29歳を対象に調査した「我が国と諸外国の若者の意識に関する調査」の「あなたは今の自国の政治にどれくらいの関心がありますか」というアンケートの結果では「非常に関心がある」「どちらかといえば関心がある」と答えた人の割合はスウェーデンが48%,日本が54%とむしろ日本の方が高いという結果が出ています.しかし「個人の力では政府の決定に影響を与えられないと思いますか」というアンケートの結果では「そう思う」「どちらかといえばそう思う」と答えた人の割合が日本は約70 %,スウェーデンは約40%でスウェーデンでは「そう思わない」「どちらかといえばそう思わない」と答える人の割合の方が高くなっています.

\section{日本とスウェーデンの「小学校社会科」の違い}
まずスウェーデンには小学校というものがなく,日本でいう小学校と中学校を合わせた9年課程の基礎学校があります.この本では小学校の高学年,基礎学校4〜6年生の社会科の教科書を比較しています.日本の社会科でいう「地理」「歴史」はスウェーデンの「社会科」には含まれておらず独立しています.スウェーデンの社会科で学習する内容は日本の中がこうで学習するような「公民」,高校で学習するような「現代社会」や「政治経済」,小中学校で学習する「道徳」といった内容になっています.

\section{まとめ}
日本とスウェーデンの若者の意識の差は「政治の関心」というレベルではなく,政治や社会に自分がどのような関わりを持つかというより根本的な意識の違いがあることを示しています.
根本的な意識の差を作っているであろう学習について今後の発表でより詳しく考えていきます.

\section{参考文献}
ヨーラン・スバネリッド,鈴木賢志・明治大学国際日本学部鈴木ゼミ編(2016)『スウェーデン小学校社会科の教科書を読む』(はじめに 訳者による解説),新評論.
  

\end{document}