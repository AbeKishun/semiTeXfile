\documentclass[a4j,11pt]{jsarticle}
\usepackage[dvipdfmx]{graphicx}

\begin{document}
\begin{flushright} %右揃え
2018年5月15日

阿部 希駿
\end{flushright}

\begin{center}
\Large{選択肢認識機能を実装したタブレット型問題演習システム}
\end{center}

\section{背景、目的}
\label{sec:mokuteki}
スマートフォンやタブレットは教育現場に導入することで学習を支援することができる要素であり,新しい学習スタイルを提供するデバイスであるが,現場での利用普及と持続可能性を考慮すると従来型の授業スタイルも無視できない.そのため従来の学習環境を尊重した上でタブレットなどを導入することを目指す. 

\section{実装環境の検討}
\subsection{持続可能な操作環境}
以前の研究で教材への追記が記憶力や理解力を高めたりが学習効果の向上が図れることが示唆されている.ソフトウェアキーボードを用いると書き込みの困難さから紙の講義ノートとの併用になると考えられる.そこで持続可能な操作環境として手書きベースの環境とする.

\subsection{持続可能な教材作成技法}
自動採点機能を持った問題演習環境を実現するためにはシステムに正誤判定を行うための正解情報を記述したりシステムのルールに従う必要がある.持続可能な教材作成のためには従来の教材作成と同等なスキルで作成可能である必要がある.

\subsection{持続可能な教材データ}
教育環境にタブレットなどが普及しても,その全てが電子端末に切り替わるわけではなく,紙と鉛筆を利用した従来のプリント環境との併用が予測される.そこで特定のシステムに依存されず可読可能であり,従来のプリント問題演習にも転用することができる形式が求められる.

\section{実装環境}
本論文では問題形式を「選択問題」とする.

\subsection{持続可能な操作環境}
従来の紙媒体での問題解答と同等の環境にするために,演習問題に対する手書き操作の他に問題の解答時に円描画を採用した.
\subsection{持続可能な教材作成技法}
JPEG形式で作成された問題ファイルに対して画像解析により選択肢の位置を検出し教材ファイルとして扱う.このことにより問題のレイアウトなどの制約がなくなる.

\subsection{持続可能な教材データ}
広く普及されているJPEGファイルを利用することで,従来のプリント教材への利用などとして利用しやすい.



\section{動作検証}
\subsection{実施方法}
10名の学生に5問ずつ問題を作成してもらい,次の項目を検証する.

・システムの処理時間

・選択肢の認識精度

・被験者の主観的な印象

\subsection{実施結果}
\subsection{システムの処理時間}
オブジェクトが増えるごとにダウンロードやアップロードにかから時間が増える.現段階では問題数を80問とした試験を行うと,試験の前後に2分以上の時間がかかってしまう.

\subsection{選択肢の認識精度}
50問中47問で94%の認識精度だった.残りの3問はフォントサイズの統一が図られてなかったことであり,被験者の理解不足であった.

\subsection{被験者の主観的な印象}
「Q1,入力方法は手書きかクリックのどちらが良いか」「Q2,手書きによる選択肢認識の印象は」「Q3,システム利用の際にメモ用紙は必要か」「Q4,教材スライド作成の制約は問題作成の妨げになったか」の4問について10名にアンケートを行った.

結果はQ1はクリックを好む被験者が6名とやや多かった.Q2は8名が違和感なく利用できたと答えた.Q3は7名の被験者がメモ用紙があったほうが便利と答えた.Q4は9名の被験者が妨げになることはないと答えた.
\section{まとめ}
現時点では問題制作においての制約は感じさせることはなかった.操作環境や問題形式のバリエーションなどは改善が必要である.

\section{参考文献}
\label{sec:bunken}
越智洋司,井手勝也“選択肢認識機能を実装したタブレット型問題演習システム”,『教育システム情報学会論文誌』Vol32(1), pp.37-47,2015.
  

\end{document}