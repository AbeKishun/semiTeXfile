\documentclass[a4j,10pt]{jsarticle}

\begin{document}
\begin{flushright} %右揃え
2018年6月26日

阿部 希駿
\end{flushright}

\begin{center}
\Large{スウェーデンの小学校社会科の教科書を読む 6}
\end{center}

\section{目的}
\label{sec:kihon}
選挙の投票率の高いスウェーデンと日本の教育の違いを社会の教科書から読み解く.

\section{政治について}
\subsection{権力}
スウェーデンの教科書には「権力は誰が持っていますか?」という質問が書いてあります.
おそらく日本でこの質問をすると政治家などの名前しか出てこないと思いますが,スウェーデンでは政治家の他にも大企業の社長や,スポーツチームの監督などが出てきます.
このように日本はスウェーデンと比べて政治と経済を分けて考えることが多いです.
\vdots

\subsection{集団}
スウェーデンでは民主的であることが特に重要視されています.そして小学校では集団を作ることは他人に影響を与えるためだという考え方をします.
例えばクラス旅行での行き先を決める時に複数の意見が出ると,同じ意見を持つ集団に分かれます.これが世論を形成することになります.

また必ず重要になってくるのが他人に意見を正しく伝えることです.
スウェーデンでは小学6年生まで成績表というものがなく,試験にも受験にも無縁な子供たちに,正しい言葉や文章を書くモチベーションを与える方法として「自分の考えを正しく伝えて,他人を説得できるようにするため」という導き方をしています.


\section{まとめ}
\label{sec:kihon}
以前から書いている通り,スウェーデンでは仕組みを教えるだけではなく具体的に社会の何に使われているか,どのように動いているかなどを教えています.またスウェーデンでは小学校の頃から他人に自分の意見を伝えること,発信することを重要視しています.

\section{参考文献}
\label{sec:kihon}
ヨーラン・スバネリッド,鈴木賢志・明治大学国際日本学部鈴木ゼミ編(2016)『スウェーデン小学校社会科の教科書を読む』第3章,新評論.
  
\end{document}