\documentclass[a4j,11pt]{jsarticle}
\usepackage[dvipdfmx]{graphicx}

\begin{document}
\begin{flushright} 
2018年5月22日%日付

阿部 希駿
\end{flushright}

\begin{center}
\Large{日本企業は「礼儀正しく時間を奪う」

マイクロソフトが働き方改革で歩んだ“地雷だらけ”の道}%タイトル
\end{center}

\section{背景,目的} %項目}
日本の労働生産性はG7の中で19年間ビリです.労働生産性とはGDPと労働人口の割り算で,サービス残業は含まれません.

アメリカと日本の忙しいの感覚はかなり異なっており,日本の忙しいは「残業」であったり「休日出勤」であったりします.
それに対してアメリカの忙しいは8時間勤務で7時間35分くらい過ぎるともう忙しいと言い始めます.つまり日本は効率が悪い働き方をしているのです.

そこでアメリカと日本の働き方の違いについて比較することで,日本の働き方の悪い点を考えていきます.

\section{アメリカと日本の「会議」} 
ある日本の企業の人が言うには会議は「偉い人が喋ってるのを聞いてメモを取る.わからないことがあれば会議が終わった後に年次の一番近い人に聞く.発言することはほとんどない」と言うことです.
しかしアメリカの企業では会議で発言しなかったら次の会議では呼ばれません.これはペナルティということではなく「理解できているからこの会議には出なくていい.時間を返す」という感覚です.

また報告・連絡・相談からくる「ほうれんそう」という言葉があります.この中の報告と連絡は「過去」,相談は「未来」のことです.日本人は「顔を合わせて伝えないと失礼だ」と考え報告や連絡に関しても時間を合わせ会議などを入れてしまいます.これは時間を無駄にするだけではなく資料を読んで考えることもできなくなってしまいます.
\section{まとめ}
日本は礼儀正しいがゆえに自分や他者の時間を奪い,判断をその場で下さなければならなかったり,勤務時間が長引くなどの問題が出ています.また報告や連絡は過去の出来事の焼き直しであるためレポートを作成するなどに時間を割くことは全く前進していないことになります.

働くにあたって何に時間を割くべきかをもう一度考え直さなければければならないと思います.

\section{参考文献}
%著者,著者“論文名”,『本の名前』Vol32(1), pp.37-47,2015.
%著者(刊行年)『本の名前』(シリーズ名)会社.
 澤円(日本マイクロソフト株式会社 テクノロジーセンター センター長 ),”日本企業は「礼儀正しく時間を奪う」マイクロソフトが働き方改革で歩んだ“地雷だらけ”の道”働き方フォーラムのスピーチ,2017/11/14,参考URL(https://logmi.jp/243422)

\end{document}