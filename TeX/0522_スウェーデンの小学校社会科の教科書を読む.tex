\documentclass[a4j,10pt]{jsarticle}

\begin{document}
\begin{flushright} %右揃え
2018年5月22日

阿部 希駿
\end{flushright}

\begin{center}
\Large{スウェーデンの小学校社会科の教科書を読む 2}
\end{center}

\section{目的}
\label{sec:kihon}
社会科の教科書の違いから日本とスウェーデンの投票率の違いの原因を考える.

\section{スウェーデンの社会科}
\label{sec:kihon}
スウェーデンではそれに加えて「法律や規則は変わるもの」と教えられています.

またスウェーデンでは社会科の時間に「考えること」に重きを置いた授業が多く行われます.例えば「原因」「社会問題」「結果」「解決策」「新しい結果」を話し合います.

\section{日本の社会科}
\label{sec:kihon}
日本では「法律や規則は守るもの」とだけ考える人が多いと感じます.

また社会科に限らず日本の授業では「暗記」に重きを置いた授業が多く,勉強する目的が「良い高校や大学に入ること」になっています.


\section{国民の学校の感じ方}
\label{sec:kihon}
内閣府が2014年に調査した「我が国と諸外国の若者の意識に関する調査」の中の「一般的・基礎的知識を身に付けるうえで、学校に通う意義がある(あった)と思う」という質問に4段階で答えました.その結果,「意義があった(ある)」と答えた人の割合が日本では28\%だったのに対しスウェーデンでは62\%と二倍以上の差が見られました.
また「どちらかというと意義があった(ある)」と答えた人を合わせると日本では70\%なのに対しスウェーデンでは91\%とかなり高い水準となっています.

\section{まとめ}
\label{sec:kihon}
日本では「暗記」が重視されている,スウェーデンでは「考えること」が重視されていることが日本とスウェーデンの一番大きな社会科の学習の差であることがわかりました.もちろん「暗記が無駄である」ということや,「スウェーデンの学習の方が優れている」ということはありません.
しかし積極的に選挙に参加したり社会について考えることができる人間を育てる面でいえば,確実にスウェーデンの教育の方が優れています.

\section{参考文献}
\label{sec:kihon}
ヨーラン・スバネリッド,鈴木賢志・明治大学国際日本学部鈴木ゼミ編(2016)『スウェーデン小学校社会科の教科書を読む』第1章,新評論.
  
\end{document}