\documentclass[a4j,10pt]{jsarticle}

\begin{document}
\begin{flushright} %右揃え
2018年5月29日

阿部 希駿
\end{flushright}

\begin{center}
\Large{日本企業は「礼儀正しく時間を奪う」

マイクロソフトが働き方改革で歩んだ“地雷だらけ”の道 2}
\end{center}
\section{目的} 
アメリカと日本の働き方の違いについて比較することで,日本の働き方の悪い点を考えていきます.

\section{席が固定されていることの問題点}
一般的に企業といえば「決められたデスクがあり,そこにPCや書類などを置く」という環境をイメージするかもしれません.
しかしこれにはコミュニケーション不全という問題点があります.
同じ席にいると同じ場所にしか行かず,自然に同じ人としか会話をしなくなります.

ここでマイクロソフトでは固定席というシステムをなくそうと考えます.


\subsection{ペーパーレスの重要性}
書類を紙で置いておくことは大きく二つのデメリットがあります.
1つ目はコスト面です.紙を印刷するためのコストもかかりますが,何よりも大量の資料によって多くの場所をとられてしまいます.
紙を保管する場所にも家賃はかかっているため大きなコストになります.
2つ目はセキュリティ面です.大量にある紙のうち何枚か抜かれたところで気づくことができません.

このような点からマイクロソフトでは資料を紙で持ち歩くことをやめ,全てスキャンをし電子データ化されました.

\subsection{出勤することは仕事ではない}

会社の中にいることは仕事をする上でのオプションの一つであり,仕事をすること自体ではありません.
ここでマイクロソフトではクラウド化が進められました.家からでもワンクリックで会社の環境にログインすることができ,会話をするときはSkypeを利用します.

また自由に出勤時間が組み換えできるようになりました.こうすることで「16時から子供のお迎えに行かなければならないから,18時からミーティングに入る」などライフスタイルに合わせた仕事を行うことができます.

\section{働き方改革によって出た結果}
この働き方改革によって,ワークライフバランスの満足度が上昇しただけではなく,残業時間や交通費を減らすことができただけではなく,女性離職率を大幅に減らすことができました.

\section{社員はサボるのではないか}
サボります.しかしサボる人たちに合わせて生産性を阻害することは得策ではありません.サボる人に合わせるのではなく仕事をする人に合わせることが重要です.そもそもサボる人は会社にいたとしてもスマホをいじったりしてきっとサボっています.

しかし重要なことはあらかじめ「何をもって成果を出した」と評価するかを明文化して合意を得ることです.サボる人に合わせるのではなく,成果を出した人を正当に評価しそちらに合わせた環境を整えていくことが一番重要です.
\section{まとめ}
\label{sec:kihon}
日本は精神論であったり古臭い考え方が成長の阻害をしている国だと感じています.
日本のマーケットは諸外国に奪われていくことが目に見えており,より効率的に,自由に,楽しく働けるようなワークスタイルを作っていくことが重要です.

\section{参考文献}
\label{sec:kihon}
 澤円(日本マイクロソフト株式会社 テクノロジーセンター センター長 ),”日本企業は「礼儀正しく時間を奪う」マイクロソフトが働き方改革で歩んだ“地雷だらけ”の道”働き方フォーラムのスピーチ,2017/11/14,参考URL(https://logmi.jp/243422)
  
\end{document}