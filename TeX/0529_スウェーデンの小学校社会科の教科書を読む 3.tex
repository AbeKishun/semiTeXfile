\documentclass[a4j,10pt]{jsarticle}

\begin{document}
\begin{flushright} %右揃え
2018年5月29日

阿部 希駿
\end{flushright}

\begin{center}
\Large{スウェーデンの小学校社会科の教科書を読む 3}
\end{center}

\section{目的}
\label{sec:kihon}
社会科の教科書の違いから日本とスウェーデンの投票率の違いの原因を考える.

\section{スウェーデンのメディア教育}
\subsection{メディアとはなにか}
日本では多くの人がメディアと聞いて思い出すものは「TV」「新聞」です.
しかしスウェーデンでは社会科の授業でニュースは「受け取るもの」ではなく,SNSなどを通して「発信するもの」として教えられます.

\subsection{メディアの利用}
小学校の時点で情報を発信することの危険性などを学び,禁止するのではなくうまく使っていくことに重きを置いています.
インターネットの利用の例として「署名を集めよう」「人を集めてデモを起こそう」「政治家に直接連絡を取ろう」などメディアとしてのインターネットを積極的に利用していくような教え方をしています.

また社会科の授業で「広告」について学びます.例えば映画に出てくる車や服などを使用する際にスポンサーの企業からお金をもらうことがあります.このような社会の仕組みについてを学ぶことで情報の発信者の意図を読み解く重要性を教えることができます.

\section{まとめ}
\label{sec:kihon}
日本のメディア教育は「有害な情報からいかに身を守るか」が重視されています.これに対してスウェーデンでは「いかに有益な情報を発信できるか」が重視されていて,メディアは民主制の道具という考え方です.

日本はソーシャルメディアを規制するのではなく,スウェーデンのように有効に利用していくことで社会の向上に役立てていくべきです.

\section{参考文献}
\label{sec:kihon}
ヨーラン・スバネリッド,鈴木賢志・明治大学国際日本学部鈴木ゼミ編(2016)『スウェーデン小学校社会科の教科書を読む』第2章,新評論.
  
\end{document}