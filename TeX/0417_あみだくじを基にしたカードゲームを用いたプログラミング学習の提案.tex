\documentclass[a4j,10pt]{jsarticle}

\begin{document}
\begin{flushright} %右揃え
2018年4月17日

阿部 希駿
\end{flushright}

\begin{center}
\Large{あみだくじを基にしたカードゲームを用いた

プログラミング学習の提案}
\end{center}

\section{目的}
\label{sec:kihon}
プログラミングをする上で難解であるプログラミング言語を図表を用いた表現に置き換えることで難解さを回避し、気軽にプログラミングの学習ができることを目指す。

\section{手段}
\label{sec:kihon}
あみだくじとカードを用いた視覚的にわかりやすいプログラミング学習をするためのゲームを用いて学習する。

\section{ゲーム概要}
\label{sec:kihon}
あみだくじに「順次」「分岐」「反復」の3種類のカードを用いる。学習者は入口と出口の色が同じになるようにカードを配置する。


\section{評価方法}
\label{sec:kihon}
大学生13名に対して授業実験を行う。「楽しさ」「易しさ」「興味」「学習効果」について悪い、やや悪い、やや良い、良いの4段階でアンケートを取る。

また学習者のゲームの成績とプログラミングの授業における成績をC、B、B+、A、A+の5段階で評価をし、相関関係を求める。


\section{評価結果}
\label{sec:kihon}
ゲームの成績と授業の成績における相関関係は明確に表れなかった。
プログラミング経験者からは「楽しい」という意見は多く聞く事ができたが、プログラミング初心者からは「難しい」という声も多かった。

\section{参考文献}
\label{sec:kihon}
望月博文,山田朗「あみだくじを基にしたカードゲームを用いたプログラミング学習の提案-csアンプラグドの応用-」,『日本教育工学会論文誌』37, p.73-76,東京学芸大学大学院教育学研究科.
  
\end{document}