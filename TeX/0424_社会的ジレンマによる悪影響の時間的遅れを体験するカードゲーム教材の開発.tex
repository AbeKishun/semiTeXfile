\documentclass[a4j,11pt]{jsarticle}
\usepackage[dvipdfmx]{graphicx}

\begin{document}
\begin{flushright} %右揃え
2018年4月24日

阿部 希駿
\end{flushright}

\begin{center}
\Large{社会的ジレンマによる悪影響の時間的遅れを

体験するカードゲーム教材の開発と評価}
\end{center}

\section{背景、目的}
\label{sec:mokuteki}
ある問題に対して個人は協力か非協力かを選べるが、個人にとっては非協力を選んだ方がよい結果となるが、全員が非協力を選んだ場合得られる利益は全員が協力を選んだ場合よりも小さくなる。これを社会的ジレンマといい代表的なものに環境問題がある。また社会的ジレンマには非協力的な行動を取った場合に悪影響が生じるまでに時間的遅れを伴うことが多い。したがってこの研究では現実世界において体験することが困難である社会的ジレンマをゲームを用いて体験させる。このことで非協力行動の悪影響を十分な形で体験させ、協力的行動をとろうとする道徳的意識を獲得させることが目的である。

この実験は以前に行われた時間的遅れについて考慮していないゲームに「時間的遅れ」に関するルールを追加したものである。


\section{ゲームのルール概要}
\label{sec:kihon}
参加者は「個人」の役割を担う。参加者は「協力的カード」「非協力カード」を用いて幸福度を最大化し、他のプレイヤーよりも上回ることを目的とする。しかし、環境に一定以上の負荷がかかると全員が負けとなる。

また時間的遅れの再現として「全員非協力の場合に遅れて悪影響が発生するルール」と「ゲーム後半では自分の子供を操作するルール」を用いた。

ゲームは「カードを出す」→「次のターンに出すカードについて話し合う」→「現在の環境負荷に応じてイベントが発生する」を1ターンとし、全部で7ターン行われる。

全員非協力だった場合の悪影響は「次のターンに環境負荷が加算されること」である。

 

\section{実験}
\subsection{実施方法}
\label{sec:kih}
32名の学生を16人ずつに割り振り「時間的遅れの要素が無いゲーム」「時間的遅れの要素のあるゲーム」に割り振った。

学生には事前質問紙に回答してもらい、ゲームを行なった後に事後質問紙に回答してもらった。
回答結果について「知識」「信頼」「道徳意識」の向上に及ぼす効果を検討した。

\subsection{評価結果}
「環境問題の知識」に関してはどちらのルールにおいても主効果が見られたが、差は確認されなかった。
「信頼」に関してはどちらのルールにおいても主効果も差も確認されなかった。
「道徳」に関しては「時間的遅れのルールがあるゲーム」に関してのみ主効果があることが確認された。





\section{事後インタビュー}
アンケート後に行ったアンケートにおいて「子供世代や世界のために自らが行動しなければならない」という「責任」について答える参加者が多く見られた。


\section{まとめ}
\label{sec:mokuteki}
本ゲームにおいて、「時間の経過」という概念を追加したことで「親になる」ことを想起し「責任感」を活性化させていた可能性があった。
また本研究の課題として「他者と協力する信頼」が獲得されなかったことがある。今後はグループ制などが追加できるかを検討し「信頼」の獲得ができるかを考えていかなければならない。

\section{参考文献}
\label{sec:bunken}
福山祐樹,森田裕介“社会的ジレンマによる悪影響の時間的遅れを体験するカードゲーム教材の開発と評価”,『日本教育工学会論文誌』Vol37(4), pp.365-374,014
  

\end{document}