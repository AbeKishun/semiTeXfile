\documentclass[a4j,10pt]{jsarticle}
\usepackage[rm,bf,compact,topmarks,calcwidth,pagestyles]{titlesec}

\begin{document}
\begin{flushright} %右揃え
2018年7月10日

阿部 希駿
\end{flushright}

\begin{center}
\Large{卒業研究進捗}
\end{center}

\section{テーマ}
\label{sec:kihon}
学習科目の全体像の把握と理解度,学習意欲の関係性

\section{本テーマについて}
研究項目が大きく分けて3つあり「全体像の把握」「理解度」「学習意欲」の3要素についてそれぞれの関係性を調べる必要がある.その中でも「理解度」と「学習意欲」の関係性については既に多くの研究で前提として言われていて,「理解度の変化によって学習意欲の変化が引き起こされる」という論文が未確認ではあるが存在すると考え,重視しなくてもよいと考える.

\subsection{「全体像の把握」と「理解度」}
仮説としては「学習内容の全体像を把握することによって,内容への理解が深まる.」というもので,本研究で一番重視している項目である.

\subsection{「全体像の把握」と「学習意欲」}
2で記述した通り,「理解度の変化によって学習意欲の変化が引き起こされる」とする場合,「全体像の理解」によって「理解度」と「学習意欲」の両方が変化する場合,「学習意欲の変化」は「理解度の変化」に起因してしまい,正確な結果が出ない可能性がある.

\section{実験について}
全体像の把握の実験に授業内容を用いた場合の問題点として「全体の学習に年単位の時間がかかること」や「授業態度の違いから起こる初期段階での理解度の差」が挙げられる.そのため本研究では理解に複数の分野の知識が必要となる項目について被験者に授業を行い実験を行う方針で考えている.

\end{document}