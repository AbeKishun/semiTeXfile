\documentclass[a4j,10pt]{jsarticle}

\begin{document}
\begin{flushright} %右揃え
2018年6月5日

阿部 希駿
\end{flushright}

\begin{center}
\Large{スウェーデンの小学校社会科の教科書を読む 5}
\end{center}

\section{目的}
\label{sec:kihon}
選挙の投票率の高いスウェーデンと日本の教育の違いを社会の教科書から読み解く.

\section{自分自身の経済について}
教科書では国内の経済格差について書かれています.「物乞いもいれば,働く必要のないほど裕福な人もいます」などはっきりとした表現で書かれています.そしてその両者がどんな夢をもって生きているかを考えさせる授業を行います.

また小学生の授業で家庭の収入と支出について考えます.家庭の収入から家賃や電気代,食費などを計算し自由に使えるお金がいくらあるかなどを考えます.

\section{税金について}
スウェーデンでは小学校で税金を収入の何割払うかなど消費税以外の税についても詳しく学びます.それと同時にその税金で行われている社会サービスについても詳しく教えられます.スウェーデンでは一般消費税が25\%ですが,福祉大国と呼ばれるほど社会福祉が充実している国です.そのような社会環境のおかげが税金に対する関心が非常に高くなっています.

\section{経済と環境について}
スウェーデンは経済の授業の中で環境問題についても学びます.例えばモノを多く消費すると,モノを多く作らなければならないため,石油などの天然資源の使用料が増えます.日本でも環境問題についてはかなり真面目に取り組んではいますが,スウェーデンでは経済の分野と結びつけることで,自分の行動で環境に影響を与えることをしっかりと意識させています.

\section{まとめ}
\label{sec:kihon}
貧困や格差について学ぶことでそれが解決すべき社会問題であることを意識させたり,環境問題について小学校の時から意識させることで国民が長期間かけて解決していくべき問題を明確にしています.


\section{参考文献}
\label{sec:kihon}
ヨーラン・スバネリッド,鈴木賢志・明治大学国際日本学部鈴木ゼミ編(2016)『スウェーデン小学校社会科の教科書を読む』第4章,新評論.
  
\end{document}