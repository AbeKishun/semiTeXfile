\documentclass[a4j,10pt]{jsarticle}
\usepackage{multirow}
\usepackage{float}
\usepackage{dcolumn} 

\begin{document}
\begin{flushright} %右揃え
2018年6月26日

阿部 希駿
\end{flushright}

\begin{center}
\Large{主体的な数学学習のための構成的アプローチに
基づく動画教材作成方法の開発と評価}
\end{center}

\section{目的}
\label{sec:kihon}
近年では動画を教材とし学習者がタブレット端末を用いて学習する例が増えている.しかしながら教室で行う授業のように密な教育支援が難しいため,より主体性が必要とされるが,講義をそのまま録画した動画教材が多く,見るだけの学習になりがちである.そのため動画教材に学習者の主体的な学習を支援するための要素を組み込む方法を開発することを目的とする.
/vdots
そのためのアプローチとして
\begin{enumerate}
\item[1.1] 学習者が実際に問題を解いたりするなどの学習活動に取り組むことができるようにする
\item[1.2] 自分自身の理解の状況を振り返らせるような問いかけを行う
\end{enumerate}
の二点から考える

\section{用意した動画教材}
高校数学の「二次曲線(楕円の方程式)」の領域の内容のものを使用する.本来の動画との変更点の主な例として1.1のアプローチより

\begin{itemize}
 \item 「ここで止めて,問題をどのように解けばいいか見当がついたら再生しよう」というテキストを挿入し,解く方略を考えさせる.
 \item 「ここで止めて,例題の答えを求められたら再生しましょう」というテキストを挿入し,例題の答えを導き出させる. 
\end{itemize}

また1.2のアプローチより
\begin{itemize}
 \item 「このように答えが求められましたか」というテキストを挿入し,自分の答えとその考え方を振り返るために問いかける.
 \item 「ここで止めて,楕円の問題は何に気をつければいいか,考えてから再生しよう」というテキストを挿入し,問題の求め方を確認させる. 
\end{itemize}


\section{実験方法}
事前調査として「高校時代の数学の成績」「この単元の学習経験」「この単元の現在の理解」の3点を回答してもらい,結果をもとにA群とB群の2グループの学力の平均が近くなるように分ける.以下の流れで実験を行う.
\begin{enumerate}
\renewcommand{\labelenumi}{\Roman{enumi}}
\item 指定された動画教材を用いて学習を行う(20 分)
\item 学習が終わった後,学習内容に関するテストを行う(解答時間 15 分)
\item テストの後,自身の学習への取り組みに関する自由記述を含む質問紙調査を行う
\end{enumerate}
またA群は講義内容を録画した従来の動画を,B群はこの研究で開発した動画を用いた.

\section{実験結果}
\subsection{事後テスト}
事後テストの結果は以下のようになった.

\begin{table}[H]
\centering
\caption{事後テストの結果}
\label{my-label}
\begin{tabular}{|c|D{.}{.}{1}|D{.}{.}{5}|D{.}{.}{5}|}
\hline
\multicolumn{2}{|c|}{問題} & \multicolumn{1}{c|}{A群} & \multicolumn{1}{c|}{B群} \\ \hline
\multirow{2}{*}{基礎} & 問1 (6点) & 5.76 & 5.90  \\ \cline{2-4} 
 & 問2 (6点) & 5.43 & 5.86 \\ \hline
\multicolumn{1}{|l|}{\multirow{3}{*}{応用}} & 問3 (6点) & 5.57 & 5.62 \\ \cline{2-4} 
\multicolumn{1}{|l|}{}  & 問4 (6点) & 3.81 & 4.81  \\ \cline{2-4} 
\multicolumn{1}{|l|}{}  & 問5 (6点) & 3.33 & 5.05 \\ \hline
\multicolumn{2}{|c|}{合計(30点)}   & 23.90 & 27.24 \\ \hline
\end{tabular}
\end{table}

\subsection{質問紙}
質問紙調査では「主体性」「知識・理解」「思考・判断」など7の観点から21の質問を行った.
その結果,21項目のうち10項目で有意差が認められた.
有意差が認められた質問の例として
\begin{itemize}
 \item この学習を通して,自ら学ぼうとする意欲が高まったと感じた
 \item この学習を通して,学習内容を十分に理解できた 
\item 何が重要な内容か,注意して学習に取り組むことができた
\end{itemize}
などがあった.

\subsection{自由記述}
\begin{itemize}
 \item 見ているだけでなく,動画を止めて自分で考えるような部分があったので,内容をより理解しやす
かった
 \item テロップが流れてくるたびに注目したり,言葉でイメージしづらいところが画像で挿入されたり
していたので,最後まで集中してみることができた 
\end{itemize}
などよい意見が得られた.

\section{まとめ}
主体的な学習を促す動画教材として,学習者に実際に問題を解いてもらうことなどが効果を発揮することが確認できた.また主体的な学習によって結果的に学習者の理解に繋がることも確認できた.

\section{参考文献}
\label{sec:kihon}
丸山浩平,森本康彦,北澤武,宮寺庸造,“主体的な数学学習のための構成的アプローチに
基づく動画教材作成方法の開発と評価”,教育システム情報学会誌vol34,pp.107\UTF{2012}121(2017)
  
\end{document}